\subsection{System modelling }
This section describes the interconnected component in a water distributed network (WDN) using Graph Theory. Furthermore thee component classes will be examined, ie. pipes, pumps and valves.


\subsection{Graph Theory}
When using graph theory as an analytical tool the incident matrix $H$ comes in handy when describing the connection between edges and nodes. The $H$ matrix is defined as follows: 

\subsubsection{The Incident Matrix}
\begin{equation}
	H_{i,j} = \begin{cases}
		-1 & \text{If the $j^{th}$ edge enters the $i^{th}$ node} \\
		0 & \text{If the $j^{th}$ edge is not connected to the $i^{th}$ node} \\
		1 & \text{If the $j^{th}$ edge s leaving the $i^{th}$ node}
	\end{cases}
\end{equation} %Description of the incident matrix


\subsubsection{The Loop Marix}
The definition of a loop is a unique route along the edges of the graph network, where all nodes are unique except the end node - In a loop the end node must also be the start node.

The loop-matrix can be calculated in one of two ways. The first one is shown in the equation below

\begin{equation}\label{eq:LoopMatrix}
	B = \begin{bmatrix}
		I & -\bar{H}_{C}^{T}\cdot\bar{H}_{T}^{-T}
	\end{bmatrix}
\end{equation}

The other method is more graphical but can be formulated as follows: 

You define your spanning tree, which is the connected graph but with no loops, i.e you can not find a route around the graph where you start and end in the same node without entering a node more than one time. 
When adding one chord to the graph, exactly one loop is created.
Then the $ B $-matrix consist of the number of loops that can be created, which is represented by the $ I $ -matrix in \cref{eq:LoopMatrix}, and the edges along each loop route. The edges has to have the same sign as in the $ H $-matrix.


\newpage
\subsection{Simplified system}
\begin{figure}[h!]
	\centering
	\includegraphics[width=0.5\textwidth]{Pictures/Graph.png}
	\caption{Graph of simplified WDN network \cite{Rathore930}}
	\label{fig:graph}
\end{figure}

When applying the rules shown above for the simplified graph model of the WDN the incident matrix in \cref{eq:H_simplified}.

\begin{equation}
	H = \begin{bmatrix}
		1 & 0 & 1 & 0 & 0 & 0 & 0\\
		-1 & 1 & 0 & 0 & 0 & 0 & 0\\
		0 & 0 & -1 & 1 & 1 & 0 & 0\\
		0 & -1 & 0 & -1 & 0 & 1 & 0\\
		0 & 0 & 0 & 0 & -1 &  0  & -1\\
		0 & 0 & 0 & 0 & 0 & -1 & 1
	\end{bmatrix}
	\label{eq:H_simplified}
\end{equation} %The incident matrix for system


The reduced incident matrix by taking an arbitrary vertex as a reference, and removing that vertex-row from \cref{eq:H_simplified}. We chose the $4^{th}$ vertex, which results in the following reduced incident matrix:
\begin{equation}
	\bar{H} = \begin{bmatrix}
		1 & 0 & 1 & 0 & 0 & 0 & 0\\
		-1 & 1 & 0 & 0 & 0 & 0 & 0\\
		0 & 0 & -1 & 1 & 1 & 0 & 0\\
		0 & 0 & 0 & 0 & -1 &  0  & -1\\
		0 & 0 & 0 & 0 & 0 & -1 & 1
	\end{bmatrix}
\end{equation}

Chords and edges of the spanning tree

\begin{equation} 
	\begin{split}
		E_{C} &= \{e_{1},e_{4}\}   \\ E_{T} &= \{e_2,e_3,e_5,e_6,e_7\}
	\end{split}
\end{equation}


\begin{equation}
	B = \begin{bmatrix}
		1 & 0 & 1 & -1 & -1 & 1 & 1\\
		0 & 1 & 0 & 0 & -1 & 1 & 1\\
	\end{bmatrix}
\end{equation}
\newpage

\subsection{Detailed system}

\begin{figure}[h]
	\centering
	\includegraphics[width=0.7\textwidth]{Pictures/GraphDetailed.png}
	\caption{Detailed graph model of WDN} 
		\label{fig:WDNDetailed}
	\end{figure}
	
	\begin{equation}
		H = \kbordermatrix{
		& e_1 & e_2 & e_3   & e_4  & e_5 & e_6  & e_7  & e_8  & e_9  & e_{10}  & e_{11}  \\	
		v_1& 1 & 0 & 0   & 0  & 0  & 0  & 0  & 0  & 0  & 0  & 0 \\
		v_2& -1 & 1 & 0  & 1  & 0  & 0  & 0  & 0  & 0  & 0  & 0 \\
		v_3& 0 & -1 & 1  & 0  & 1  & 0  & 0  & 0  & 0  & 0  & 0 \\
		v_4& 0 & 0  & -1 & 0  & 0  & 0  & 0  & 0  & 0  & 0  & 0 \\
		v_5& 0 & 0  & 0  & -1 & 0  & 1  & 1  & 0  & 0  & 0  & 0 \\
		v_6&0 & 0  & 0  & 0  & -1 & -1 & 0  & 1  & 0  & 0  & 0 \\
		v_7& 0 & 0  & 0  & 0  & 0  & 0  & 0  & 0  & -1 & 0  & 0 \\
		v_8& 0 & 0  & 0  & 0  & 0  & 0  & -1 & 0  & 1  & -1 & 0 \\
		v_9& 0 & 0  & 0  & 0  & 0  & 0  & 0  & -1 & 0  & 1  & -1 \\
		v_{10}& 0 & 0  & 0  & 0  & 0  & 0  & 0  & 0  & 0  & 0  & 1 \\
	}
	\end{equation}	
	
\begin{equation*} 
	\begin{split}
		E_{C} &= \{e_{2},e_{6}\}   \\ E_{T} &= \{e_1,e_3,e_4,e_5,e_7, e_8, e_9, e_{10} , e_{11}\}
	\end{split}
\end{equation*}	
	
	By removing $v_{6}$ from $H$ we get $\bar{H}$ which results in:
	\begin{equation}
		\bar{H} = \kbordermatrix{
		& e_1 & e_2 & e_3   & e_4  & e_5 & e_6  & e_7  & e_8  & e_9  & e_{10}  & e_{11}  \\	
		v_1& 1 & 0 & 0   & 0  & 0  & 0  & 0  & 0  & 0  & 0  & 0 \\
		v_2& -1 & 1 & 0  & 1  & 0  & 0  & 0  & 0  & 0  & 0  & 0 \\
		v_3& 0 & -1 & 1  & 0  & 1  & 0  & 0  & 0  & 0  & 0  & 0 \\
		v_4& 0 & 0  & -1 & 0  & 0  & 0  & 0  & 0  & 0  & 0  & 0 \\
		v_5& 0 & 0  & 0  & -1 & 0  & 1  & 1  & 0  & 0  & 0  & 0 \\
		v_7& 0 & 0  & 0  & 0  & 0  & 0  & 0  & 0  & -1 & 0  & 0 \\
		v_8& 0 & 0  & 0  & 0  & 0  & 0  & -1 & 0  & 1  & -1 & 0 \\
		v_9& 0 & 0  & 0  & 0  & 0  & 0  & 0  & -1 & 0  & 1  & -1 \\
		v_{10}& 0 & 0  & 0  & 0  & 0  & 0  & 0  & 0  & 0  & 0  & 1 \\
		}
	\end{equation}
	
Furthermore we can define the open-node matrix $F$ and tank matrix $G$:
	
	\begin{equation}\label{eq:ONandTankMatrix}
		F = \kbordermatrix{
			&d_{f_1}&d_{f_2}&d_{f_3}&d_{f_4}\\
		d_1	& 1 & 0 & 0 & 0\\
		d_2	& 0 & 0 & 0 & 0\\
		d_3 & 0 & 0 & 0 & 0\\
		d_4 & 0 & 1 & 0 & 0\\
		d_5 & 0 & 0 & 0 & 0\\
		d_6 & 0 & 0 & 0 & 0\\
		d_7 & 0 & 0 & 1 & 0\\
		d_8 & 0 & 0 & 0 & 0\\
		d_9 & 0 & 0 & 0 & 0\\
		d_{10}& 0 & 0 & 0 & 1 \\
			},
	\qquad
		G = \kbordermatrix{
			&d_{\tau_1}\\
			d_1& 0\\
			d_2& 0\\
			d_3& 0\\
			d_4& 0\\
			d_5& 0\\
			d_6& 1\\
			d_7& 0\\
			d_8& 0\\
			d_9& 0\\
			d_{10}& 0 \\
			}
	\end{equation}

with their reference-respective equivalents given by:

\begin{equation}\label{eq:FbarGbar}
	\bar{F} = F \setminus F_{6\star} \wedge \bar{G} = G \setminus G_{6\star}
\end{equation}

where the notation $X_{6\star}$ denotes the entire 6th row\footnote{Correspondingly, $X_{\star6}$ would denote the entire 6th column} of the matrix $X$ and $\setminus$ is the set relative complement operator. These matrices map demands at their respective nodes into the vector of total demands in the system $d \vee \bar{d}$.


