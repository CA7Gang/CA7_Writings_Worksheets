This section is concerned with the appendices of the WorkSheets

\subsection{PumpCoefficients}

A model of the pump is based on the following relationship between the pressure drop over the pump and the pump speed, flow through the pump and the pump coefficients.

The full pump model:
\begin{equation} \label{eq:pumpmodel}
		\Delta p_{pump}  = a_0  \cdot \omega^2 + a_1 \cdot q \cdot \omega + a_2 \cdot q^2
\end{equation}

The pump coefficients are estimated from the pump performance curves from the data-sheets. These describe the exact relationship described in \cref{eq:pumpmodel}.

The preassure drops were read at the following flows:

\begin{equation}
	q = \begin{bmatrix}
		0 & 0.5 & 1 & 1.5 & 2 & 2.5 & 3 & 3.5 
	\end{bmatrix} 
\label{eq:pump_q}
\end{equation}

At these flow speeds the pressure drop across the pump were read from the y-axis at three different pump speeds, namely 80\%, 69\% and 48\%. The recorded preassure drops were the following:

\begin{equation}
	p =  \begin{bmatrix}
		p_{80}\\
		p_{69} \\
		p_{48}
	\end{bmatrix}  = 
	\begin{bmatrix}
		5.57 & 5.85 & 5.9 & 5.7 & 5.1 & 4.253.3 & 3.3 & 2.4 \\
		4.1 & 4.2 & 4.2 & 4 & 3.55 & 3 & 2.3 & 1.55 \\
		1.75 & 1.85 & 1.7 & 1.3 & 0.95 &  &  &  
	\end{bmatrix} 
\label{eq:pump_p}
\end{equation}

Only 5 pressure readings were made from the 48\% pump speed.

The coefficients are found by the least square solution to:

\begin{equation}
	Ax = b
\end{equation}
where 
\begin{equation}
	A = \begin{bmatrix}
			\omega^2 & q \cdot \omega & q^2
			\end{bmatrix}
\end{equation}
and b = p.

The solution yields the following coefficients:
\begin{itemize}
	\item test
	\item test 2
\end{itemize}









\section{Legend for model}

Model:
\begin{equation}\label{eq:NonLinearModelWithTank}
	\Phi\mathcal{J}\Phi^T \dot{q} = -\Phi\Big(\lambda(q_n)+\mu(q_n,OD)+\alpha(q_n,\omega)\Big) + \Psi(\bar{h}-\mathbf{1}h_0) + \mathcal{I}(p_{\tau}-\mathbf{1}p_0)
\end{equation}

where $p_{\tau}$ evolves according to:

\begin{equation}\label{eq:TankDynamics}
	\dot{p}_{\tau} = - \mathcal{T} \dot{d}_{\tau}, \ \mathcal{T} = diag(\tau_i)
\end{equation}

and the matrices $\Phi, \Psi, \mathcal{I}$ are defined as:

\begin{equation}\label{eq:NonLinearModelMatrices}
	\Phi \triangleq 
	\begin{bmatrix} 
		I & -\bar{H}_C^T\bar{H}_T^{-T} \\ 0 & \bar{F}^T\bar{H}_T^{-T} \\ 0  & \bar{G}^T\bar{H}_T^{-T} \\ 
	\end{bmatrix}
	, \qquad
	\Psi \triangleq
	\begin{bmatrix}
		0 \\ \bar{F}^T \\ \bar{G}^T
	\end{bmatrix}
	, \qquad
	\mathcal{I} \triangleq
	\begin{bmatrix}
		0 \\ 0 \\ I
	\end{bmatrix}
\end{equation}

Contained in $\Phi$ and $\Psi$ are:
\begin{center}
	\begin{tabular}{l p{10cm} l}
		$F$ & Open-node matrix - Extracts the pumps and consumer nodes & Dim = n x e \\
		$G$ & Tank-node matrix/vector - Extracts the tank node. & Dim = n x 1 (one tank node) 
	\end{tabular}
\end{center}

$ \mathcal{J}$ is the mass inertia of the water in all pipes. Dim =  m x m

$\lambda (q_n)$ is the pressure drop across all components due to pipe friction [\si{Pa}]
\begin{equation}
	\lambda(q)  =	\Big(f \cdot \frac{8\cdot L\cdot q^{2}}{\pi^{2}\cdot g \cdot D^{5}} + k_{f}\cdot \frac{8\cdot q^{2}}{\pi^{2}\cdot g \cdot D^{4}}\Big)\cdot g \cdot \rho
\end{equation}


\begin{center}
	\begin{tabular}{l p{10cm} l} 
		f & friction constant &	[$ \cdot $]		\\
		L & length of pipe &	[m]		\\
		g & gravitational acceleration & [$ \frac{m}{s^2} $]\\
		D & diameter of pipe & [m]			\\
		$ \rho $ & density of water & [$ \frac{kg}{m^3} $]	
	\end{tabular}
\end{center}

$\mu(q,OD)$: is the pressure drop across all components in the system due to valve friction [\si{Pa}]

\begin{equation}\label{eq:ValvePressure}
	 	 \mu(q_n,OD) = \frac{1}{K_{valve}(OD)^2} \cdot |q|\cdot q
\end{equation}

\begin{center}
	\begin{tabular}{l p{10cm} l} 
		f & friction constant &	[$ \cdot $]		\\
		L & length of pipe &	[m]		\\
		g & gravitational acceleration & [$ \frac{m}{s^2} $]\\
		D & diameter of pipe & [m]			\\
		$ \rho $ & density of water & [$ \frac{kg}{m^3} $]	
	\end{tabular}
\end{center}

$\alpha(q,\omega)$: Pressure drop across the pump ($\Delta p_p$) [\si{Pa}]
% F: Open-node matrix - Extracts the pumps and consumer nodes. Dim = n x e
%G: Tank-node matrix/vector - Extracts the tank node. Dim = n x 1 (one tank node)


Pipe:
\begin{center}
	\begin{tabular}{l p{10cm} l}
		
		$\Delta{p_{k}}$ & The differential pressure across the $k^{th}$ component & [\si{Pa}]\\ 
		${J_{k}}$ & Is the mass inertia of the water in the $k^{th}$ pipe & [\si{kg}/\si{m^{4}}] \\
		$q_{k}$ & is the flow of water trough the $k_{th}$ pipe & [{\si{\meter\cubed}/\si{s}}] \\
		$\mathrm{\lambda_{k}}(q_{k})$ & is the drop in pressure due to friction in the $k^{th}$ pipe & [\si{Pa}] \\
		$\mathrm{\Delta{z_{k}}}$ & is the drop in pressure due to geodesic level & [\si{Pa}]\\
	\end{tabular}
\end{center}


\begin{equation}
	\Delta{p} = \mathcal{J}\dot{q} + \lambda(q) + \Delta z
\end{equation}

\begin{center}
	\begin{tabular}{l p{10cm} l}
		$\mathrm{\lambda}(q)$ & is the drop in pressure due to friction in all pipes & [\si{Pa}] \\
		$ \mathcal{J}$ & Is the mass inertia of the water in all pipes & [\si{kg}/\si{m^{4}}] \\
		$\Delta z$ & is the drop in pressure due to geodesic level & [\si{Pa}]\\
	\end{tabular}
\end{center}


Valve:

$\mu(q,OD)$: Pressure drop across a valve

Pump:

$\alpha(q,\omega)$: Pressure drop across the pump ($\Delta p_p$)


\textbf{WDN stuff:}
$H_T \bar{H_T}^-1$: Can be used to


