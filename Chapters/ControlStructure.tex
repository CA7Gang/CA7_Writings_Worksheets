This section is concerned with the control structure of the Water network

\section{Linearised model}
The linearised model of the system can be expressed on the standard state space form given in \cref{eq:StateSpace}
\begin{equation}\label{eq:StateSpace}
	\begin{split}
	\dot{x} = Ax + Bu \\
	y = Cx
	\end{split}
\end{equation}
The inputs to our system will be the pump speeds. 
The outputs will be the measured pressure at the tank node and the flows of the pumps. This yields
\begin{equation}\label{eq:StateSpaceInputsOutputs}
	\begin{split}
		u = \begin{bmatrix} \omega_1 \\ \omega_2	\end{bmatrix} \\
		x = \begin{bmatrix} q_c \\ d_f \\ d_{\tau}	\end{bmatrix} \\
		y = \begin{bmatrix} d_1 \\ d_{10} \\ p_{\tau}	\end{bmatrix} \\
	\end{split}
\end{equation}





\subsection{Choice of state variables}

\subsection{Choice of inputs and output variables}

\subsection{Cascaded control}

\subsection{Influence of pump delay}
The pumps of the WDN system introduces a delay of approximately 5 seconds. According to \cite{Skogestad2005} (pp. 182-183), 
a time delay $\theta$ limits the possible bandwidth of a system to be 

\begin{equation}\label{eq:BWdelay}
	\omega_c < \frac{1}{\theta}
\end{equation}

This naturally sets a limitation as to how fast the inner loop of the control structure can be. As the inner loop sets a limitation of how fast the outer loop can be, i.e. 5-10 timer slower than the inner loop.

