The following document details the work done by the CA 733 group at Aalborg University's M.Sc. Control and Automation program during the autumn of 2021. The work addresses modelling and control of a small-scale water distribution network (WDN), and addresses, among other things, cascaded control, velocity-form formulations of the classic Linear-Quadratic Regulator (LQR), root locus design, consumer demand estimation via Kalman filtering, and modelling of packet loss effects in geographically distributed networks.

The document is structured as follows: In \cref{sec:GraphTheory}, we introduce the basic graph theory that is a prerequisite for modelling of WDNs. In \cref{sec:ComponentModels} we then introduce models of the components that describe the behaviour of the edges in a graph-based WDN model. Development of the full model of the WDN then follows in \cref{sec:SystemModel}. After developing the model, we detail our control structure and linearise the non-linear system model in \cref{sec:ControlStructure}. Here we also design the PI controllers that govern the inner loop. A review of optimal control then follows in \cref{sec:LQR}, where we develop the velocity-form LQR (VF-LQR) used to control the tank level in the outer loop via an equivalent pressure reference. In \cref{sec:KalmanFilter}, this is followed by a review of Kalman filtering in both its time-varying and LTI form, and a scheme for modelling consumer-induced disturbances by a truncated Fourier series. Finally, \cref{sec:PacketLoss} models the stability impact of packet loss in a hypothesised control structure over a wireless medium, and \cref{sec:Results} presents the experimental validation of the proposed control structure and the expected impact of packet loss.